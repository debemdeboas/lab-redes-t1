% \begin{figure}[H]
%     \centering
%     \includegraphics[scale=0.6]{ES - Diagrama casos de uso.png}
%     \caption{Diagrama Casos de Uso}
%     \label{fig:Diagrama casos de uso}
% \end{figure}

% A \figureautorefname{ \ref{fig:PIM}} apresenta o método \lstinline{Point.calculate_quadrant()} que, a partir das entradas \lstinline{x} e \lstinline{y} presentes na instância da classe \lstinline{Point}, retorna o quadrante correspondente daquele ponto. Se o ponto estiver em (0, 0), o retorno é "Origem".

% \chapter{Modelos}

% \section{Modelo Inclusivo}

% Um modelo inclusivo é um modelo que foi esboçado rapidamente utilizando ferramentas simples como um quadro branco ou cartões \cite{AMDD}.
% Um exemplo desse modelo é o diagrama de casos de uso.


% \begin{enumerate}
%   \item O nome dos participantes está visível no diagrama. \checkmark{}
% \end{enumerate}